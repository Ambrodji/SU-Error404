\documentclass[12pt, a4paper]{article}
%-------------------------------------------------
%       PACKAGES
%-------------------------------------------------
\usepackage[utf8x]{inputenc}          % Allows the user to input accented characters directly from the keyboard
\usepackage[T1]{fontenc}              % Oriented to output, that is, what fonts to use for printing characters 
\usepackage[danish]{babel}            % The language, you are writing in - you should be able to change to english
\usepackage{color}
\usepackage{caption}                  % You can create captions for your tables, pictures etc. AND YOU WANT TO DO THAT
\usepackage{wrapfig}                  % Allows in-line images if needed
\usepackage{hyperref}                 % Allows you to create hyperlinks within the document
\usepackage{url}                      % Enables typesetting of hyperlinks

\usepackage{amsmath,amsfonts,amssymb} % Math in your tex 
\usepackage{mathtools}                % Math in your tex 

\usepackage{csquotes}                 % Provides advanced facilities for inline and display quotations
\usepackage[titletoc]{appendix}       % Names appendices "Appendix A" instead of just A in Contents

\usepackage{pdfpages}                 % We can now import pdf files to our tex file - win!
\usepackage{graphicx}                 % We can now import pictures - uhlalalah!
\usepackage{float}                    % Pictures stay were i tell them

\usepackage{algorithm}                % Beautiful algorithms
\usepackage{algpseudocode}            % Beautiful pseudocode

\usepackage{natbib}                   % Bibliography stuff

\usepackage{enumitem}                 % Control the layout of enumerate, itemize and description
\usepackage{fancyhdr}                 % Fancy header and footer
\usepackage{booktabs}                 % Nice tables
\usepackage{changepage}               % Change the page layout in the middle of a document
\usepackage{tipa}                     % Fonts

\usepackage{listings}                 % Typesetting programs
\usepackage{color}                    % Color management
\usepackage{wallpaper}
\usepackage{palatino}

\setlength\parindent{0pt}             % Removes the indent on first line of each each section

%-------------------------------------------------
%       FSHARP LISTINGS
%-------------------------------------------------

% Kan måske erstates med \usepackage{clrscode3e}

\definecolor{bluekeywords}{rgb}{0.13,0.13,1}
\definecolor{greencomments}{rgb}{0,0.5,0}
\definecolor{turqusnumbers}{rgb}{0.17,0.57,0.69}
\definecolor{redstrings}{rgb}{0.5,0,0}

\lstdefinelanguage{FSharp}
                {morekeywords={let, new, match, with, rec, open, module, namespace, type, of, member, and, for, in, do, begin, end, fun, function, try, mutable, if, then, else},
    keywordstyle=\color{bluekeywords},
    sensitive=false,
    morecomment=[l][\color{greencomments}]{///},
    morecomment=[l][\color{greencomments}]{//},
    morecomment=[s][\color{greencomments}]{{(*}{*)}},
    morestring=[b]",
    stringstyle=\color{redstrings}
    }

\lstset{language=FSharp, breaklines = true, stepnumber = 1, firstnumber = 1, numbers = left}              % set \lstlisting to always parse text as F# code

%-------------------------------------------------
%       DOCUMENT CONFIGURATIONS
%-------------------------------------------------
\fancyhf{}
\hypersetup{colorlinks=false,hidelinks, citecolor=black, urlcolor=black}
\DeclarePairedDelimiter{\ceil}{\lceil}{\rceil} % Math in your tex 
\DeclarePairedDelimiter\floor{\lfloor}{\rfloor} % Math in your tex


%%  Begin document
%%  ==================================================================
\begin{document}

%   Forside
%   ==================================================================
    \includepdf[pages={-}]{forside.pdf}

%%%%%%%%%%%%%%%%%%%%%%%%%%%%%%%%%%%%%%%%%%%%%%%%%%%%%%%%%%%%%%%%%%%%%
\section{Implementering af opgaver}
Undersektionen her, markeret med punkter er vores idéer fra sidste opgave, som vi havde planlagt. Dog vil de tilstående underkommentarer være hvor langt vi er kommet med de forskellige implementationer og iterationer. \\ Det er vigtigt at nævne at diverse punkter stadig kan være under konstruktion til sent i projektet. Dette skyldes projektets iterative natur samt relevante krav, til hvilke og måske mange funktioner, vi gerne vil have de forskellige punkter indebærer. \newline \\
% I skal implementere de opgaver som I har planlagt for denne periode. I kan oprette opgaver som issues i GitHub og bruge dem til at koordinere arbejdet hen imod næste delaflevering; se f.eks. milestone- og assignee-attributterne. Det er op til jer i gruppen hvordan I vil gennemføre opgaverne.
Vores arbejdsform og uddeling af opgaver forgår sig således at, rapporter bliver skrevet online hvor alle kan deltage. Det vigtiste er at vi er samlet om koden og alle får indblik i hvordan systemet virker, så vi ikke får undtagelser nogle af os ikke har viden om.
\subsection{Funktionelle krav}
%Funktionelle krav er en betegnelse der dækker over kravene til selve applikationens funktionalitet - Der fokuseres dermed på \emph{hvad} der skal kunne gøres, og ikke \emph{hvordan}. De opstillede funktionelle krav kan ses nedenfor:
\begin{itemize}
  \item \textbf{Dynamisk webplatform} - %Applikationen skal fungere som en web-applikation, der kan tilgås fra moderne webbrowsere (Google Chrome, Safari, Firefox, osv)
  Webplatformen er skrevet i Java og kører på en Tomcat 7 server som et WAR (\textbf{W}eb Application \textbf{Ar}chive). Serveren benytter SQLite som database, og kommunikerer med 'spil' skrevet i F\# ved hjælp af Json arrays skrevet til stdin/stdout. Frontenden er designet så brugervenligt som muligt med brug af Twitter Bootstrap API'et, og efterfølgende JavaScript. Dette sikrer at alle moderne browsere med JavaScript kan benytte applikationen uden installation af tredje parts plugins eller tilsvarende.
  
  \item \textbf{Rolle-opdeling} - %Det skal være muligt at oprette brugere i systemet, med en af følgende roller: \emph{administrator}, \emph{lærer} eller \emph{elev}. (Beskrivelse af de forskellige rollers tilgængelige funktionalitet beskrives løbende i efterfølgende krav)
  Denne funktionalitet er stadig under konstruktion.
  
  
  
  \item \textbf{Brugeradministration} - %Administratorer skal være i stand til at oprette/redigere og slette brugere.
  Denne funktionalitet er stadig under konstruktion - Der er dog på nuværende tidspunkt implementeret funktionalitet som danner fundamentet for videre udvikling: Det er muligt at tilføje og slette brugere fra databasen idet de respektive Java-klasser er blevet implementeret. Der fokuseres derfor på design af administrations-delen, således at frontenden kan blive koblet sammen med backenden.
  
  
  \item \textbf{Indhold} - %Det skal være muligt for elever at deltage i 'spil', der består af en række spørgsmål eleverne skal svare på. Der skal som minimum være 1 spil tilgængeligt, som dels fungerer som et \emph{proof of concept}, og dels som en skabelon som lærere kan benytte sig af til oprettelse af nye spil
  Vores første spil er funktionelt i den forstand at det virker som forventet når det eksekveres fra terminalen. Endvidere fungerer serveren, idet det er muligt at se sider den genererer. Serveren er også i stand til at eksekvere et spil, og hente spørgsmål fra spillet (gennem Json strings) og vise det på en plain HTML side. Funktionaliteten, der skal præsentere brugeren for selve spørgsmålene og give mulighed for at svare, er dog ikke implementeret endnu. Dette planlægges til næste fase.
  
  \item \textbf{Spil skabelon} - %Spillet skal kunne generere pseudo-tilfældige aritmetiske udtryk (med plus, minus, division og multiplikation) der skal evalueres af eleven. Svarenes korrekthed skal valideres i F\#.
  Validationen her vil foregå primært ud fra strings, programmet vil generere et string som den selv kan evaluerer i koden. Herefter skal brugeren løse opgaven ved også at give et string som input til koden. Sammenligning imellem strings forekommer, hvorved programmet kan gyldiggøre om det er den korrekte løsning til opgaven.
  
  
  \item \textbf{Statistik} - %Brugere med rollen 'Administrator' eller 'lærer' skal kunne se statistik for elevers deltagelse og resultater fra diverse spil.
  Denne funktion er stadig under konstruktion.
  
  \item \textbf{Opdatering af indhold} - %Brugere med rollen 'lærer' skal være i stand til at uploade nye 'spil' systemet.
 På nuværende tidspunkt er det spil som eksekveres \emph{hardcoded} i serveren. Det forventes, at serveren i stedet for tillader dynamisk konfiguration efter næste fase, og dermed, at administratorer er i stand til at uploade nye spil.
  
  \item \textbf{Feedback} - %Brugere med rollen 'lærer' skal kunne give skriftlig feedback for et afsluttet spil til eleven der udført spillet. Elever skal kunne se en historik og spil de har deltaget i, samt eventuel feedback for disse.
  Denne funktion er stadig under konstruktion.
  
  
  \item \textbf{Brugerinddeling} - %Elever skal kunne knyttes til lærere, således at de får adgang til et katalog af spil der er bestemt af læreren.
  Indtil videre er hele konceptet omkring 'hold' defineret i databasen samt respektive java-klasser. Selve logikken og håndtering af hold er dog hverken implementeret i backend- eller frontend-delen. Det forventes, at arbejdet på implementation af dette starter næste fase.
  
\end{itemize}

\subsection{Ikke-funktionelle krav}
%I modsætning til de funktionelle krav, beskæftiger ikke-funktionelle krav sig med \emph{hvordan} tingene skal gøres. Endvidere dækker de over krav i forbindelse med \emph{usability}. De opstillede ikke-funktionelle krav kan ses nedenfor:
\begin{itemize}
    \item \textbf{Effektivitet} - %Systemet skal være skalerbart og fleksibelt i en sådan grad, at en gennemsnitlig bærbar computer med serverapplikationen kørende, skal kunne understøtte 100 samtidigt forbundne klienter og samtidigt have en svartid på under 2 sekunder per besvaret spørgsmål.
    Idet systemet på nuværende tidspunkt er i et stadie, hvor det kun fungerer som simpel prototype, vurderes det, at effektivitetskrav først blive relevante at undersøge når implementation er mere fremskreden.
    
    \item \textbf{Tilgængelighed - Elev} - %En gennemsnitlig bruger med rollen "elev" skal kunne gennemføre et "spil" korrekt efter 5 minutters introduktion til systemet.
    Idet systemet på nuværende tidspunkt er i et stadie, hvor det kun fungerer som simpel prototype, vurderes det, at tilgængelighedskrav først blive relevante at undersøge når implementation er mere fremskreden.
    \item \textbf{Tilgængelighed} - %Administrator} - En gennemsnitlig bruger med rollen "administrator" skal være i stand til at oprette/slette 
    Databasen og java-klasserne er defineret således at det kan lade sig gøre. Indtil videre mangler der dog et administrativt brugerinterface som skal designes i frontenden først, og derfor mangler logikken der håndterer brugerrettigheder også på nuværende tidspunkt.
    
    \item \textbf{Sværhedsgrad} - %Det skal være muligt at definere sværhedsgrad for et givent spørgsmål, således at det er muligt at præsentere brugeren for et spil der gradvist bliver sværere og sværere.
    Vores spil har allerede sværhedsgrad implementeret, det er afhængigt af tal imellem 1 - 5 som vil blive indtastet i kommando-prompten.Programmet eller spillet tager denne int som et argument og kører spillet ud fra det.
    
    
\end{itemize}





% Redegør kort for hvordan I løbende har planlagt og koordineret arbejdet i denne iteration (eller iterationer, hvis I har planlagt sådan) i gruppen. Giv en kort sammenfatning af hvilke use cases I har færdigimplementeret i denne delaflevering.


Vi har haft aftaler hvor vi samles og koder sammen, højeste prioritet var at få den java baseret server til at kunne godtage out og input fra fsharp programmer.
Næste prioritet var at lave et simpelt spil som kunne virke på serveren, for at ---

Design er også en vigtig del for brugervenligheden af hjemmesiden, vi har kodet et henholdsvis overskueligt brugergrænseflade --


\section{Afprøvning}

%I skal implementere unittests der dækker de metoder I har implementeret op til denne delaflevering. I skal i denne iteration også have brugt en test-først tilgang til implementeringen af mindst én af jeres use cases, dvs. skrevet unittest før I har skrevet programkode der implementerer funktionaliteten. Redegør for i hvilket omfang I har brugt en sådan test-først tilgang og for jeres oplevelser (har I erfaret problemer, fundet det nyttigt, etc.).

Unittests er meget vigtigt for et program med flere linjers kode, indtil videre har det været svært at implementere unittests da vores kode langt fra kommer op på et problematisk eller uoverskueligt niveau. Vi har dog udført en unittest til ét af vores programkoder, som præsenterer en "test-først" tilgang. Vi har valgt at unitteste vores skabelon, da denne vil blive brugt i alle spilene, som skal køre på dette system. Da det ville kræve noget der hedder Dependency injection, for at unitteste på det vi skriver til konsolen, som ikke er lige til, så har vi valgt at skrive noget kode ind i vores skabelon. Denne kode er kun til at unitteste med og bliver fjernet senere. På denne måde er vores unittest ikke fuldendt, men vi har givet et bud på hvordan vi kan teste alle vores spil på denne måde. Se vores unit tests her: \ref{bilag:Unit-Test} \\

%I skal også inkludere accepttest for de use cases som I har implementeret (eller er begyndt at implementere) op til denne delaflevering.

%I skal inkludere en testrapport i bilag som viser resultatet af både unittest og accepttest.

Da det ikke har været muligt for os at unittest vores main funktion, så har vi valgt at teste den manuelt. Se de manuele tests her: \ref{bilag:Manuel-Test}

\section{Design}

Vores design er bygget mest på tanken om fleksibilitet. Vores system er bygget på en måde hvor spil, hjemmesiden og databasens design er forholdvis uafhængige af hinanden. Til Teacher's Assistents, som udvikler spil til systemet sætter vi kun to krav, hvor det sidste måske kan ekskluderes på et senere tidspunkt i udviklings fasen. Det første krav er at de følger den skabelon, som vi har givet til spillets struktur. Dette sikrer, at serveren kan interagere korrekt med programmet. Strukturen kan ses her: [henvisning]. Det andet krav er at spillet skal være kompileret til én enkel exe fil, før den uploades - også af hensyn til serveren. Udover dette har de stor fleksibilitet i hvordan de kan opbygge deres spil. \\

Serveren eksekverer diverse spil (kompilerede F\# programmer) ved at kalde programmet med nogle parametre der overholder den standard vi har implementeret i skabelonen. Det er dog et krav fra serverens side, at der er installeret et F\# miljø på computeren den køres på. Se vores template her: \ref{bilag:Template}

På serveren har vi en database som tager imod ét til to JSON argumenter fra spillets første funktion, getQuestion(x). Disse to argumenter er en lang string, som skal indeholde en start og slut curly braket. Inden i disse curly brakets skal der være argumentet ''question'' og hvis spillet er designet på en sådan måde, at det ikke er muligt at evaluere svaret ud fra en string. Så kan denne også indeholde argumentet ''answer''. Denne string skal overholde JSON standarten. \\

Når brugeren har svaret på spørgsmålet, så bliver svaret og spørgsmålet sendt til spillet som argumenter til funktionen evalAnswer(question, answer). Dog sker dette kun hvis systemet ikke allerede har svaret, hvor systemet selv har et stykke kode, som evaluerer og brugerens svar er lig med spillets svar. Dette giver spil udvikleren fleksibiliteten til at udregne svaret først eller gøre det senere. evalAnswer's funktion er netop at udregne om spørgsmålet givet til brugeren, og svaret givet af brugeren er det samme. Udfra dette returneres der en boolean til konsolen. \\

Når et spil uploades til systemet, så kan uploaderen skrive: \\
\begin{itemize}
  \item Navn på spillet
  \item Navn på uploaderen
  \item Hvor mange sværhedgrader spillet har
  \item Hvilket type spillet er: Multiple choice, Spørgsmål/Svar eller andre typer spil, som kan tilføjes til systemet
\end{itemize}

\subsection{SOLID}
For at undgå de mest gængse faldgruber har vi forsøgt at vurdere systemet ud fra en model beskrevet i bogen om agil udvikling:

\begin{enumerate}
    \item[] \textbf{Rigidity} - Vores projekt er det modsatte af Rigid, da det let kan udvides på, både på server siden og spil siden.
    \item[] \textbf{Fragility} - Projektet er ikke skrøbeligt, da det netop er bygget på fleksibilitet, så vi har været meget opmærksomme på at det skal kunne udvides på en let måde, hvor andet ikke bliver brudt. Et eksempel er f.eks. brugen af Json til kommunikation mellem server og F\# spil. Brugen af Json gør nemlig, at det er let at tilføje flere "felter" til dataen der bliver overført.
    \item[] \textbf{Immobility} - Systemet har en række afhængigheder som er afgørende for korrekt programkørsel, som f.eks. adgang til et F\# runtime environment, samt en  JRE og Tomcat server. Vi har dog arbejdet for at eliminere for mange unødvendige afhængigheder. Projektet benytter derfor Apache Maven som \emph{build manager}. Dette resulterer i, at en given udvikler nemt kan hente projektet fra GitHub og automatisk få hentet og konfigureret alle libraries der er nødvendige for at køre projektet. Endvidere har der været fokus på enkelthed og simplicitet - Vi har derfor valgt at bruge en SQLite database, da vi dermed undgår at være afhængige af en decideret database server. På nuværende tidspunkt vurderes det dermed, at systemet ikke viser tegn på immobilitet.
    \item[] \textbf{Viscosity} Idet projektet befinder sig i et tidligt stadie, har det nogengange været nødvendigt at benytte midlertidige "hacks" i forbindelse med udvikling og testing, da grundlæggende funktionalitet har manglet. Vi vurderer dermed, at projektet på nuværende tidspunkt befinder sig i en risikozone på dette punkt, omend vi forventer at problemet i høj grad mindskes i næste iteration.
    \item[] \textbf{Needless complexity} - Vi mener ikke at vores har unødvendig complexity, da vi både har tænkt fremad, i hvordan vores system skal kunne overleve fremtidige ændringer, og vi har på samme tidspunkt taget brug af disse features i vores to første spil. Dvs. at disse ting har været en nødvendig kompleksitet.
    \item[] \textbf{Needless repetition} - Vi vurderer, at projektet er i et for tidligt stadie til at kunne analysere dette parameter nøjagtigt, og det udelades derfor.
    \item[] \textbf{Opacity} - Vi vurderer, at projektet er i et for tidligt stadie til at kunne analysere dette parameter nøjagtigt, og det udelades derfor.
\end{enumerate}



%Redegør for jeres design i denne delaflevering. Beskriv først de objekter, og deres indbyrdes ansvar og samarbejde, som er centrale i implementeringen af de vigtigste use cases. Brug UML-diagrammer hvis det er nødvendigt.

%Redegør dernæst for de væsentligste abstraktioner og designmønstre i jeres design. I skal forklare hvordan de abstraktioner og designmønstre har gjort programmet lettere at forstå (hvad har I gjort for at begrænse unødig kompleksitet), hvor fleksibelt jeres program er for ændringer og videreudvikling (dvs. hvor ændringer kan begrænses til få steder), mv. Forklar jeres valg med henvisning til SOLID-principperne fra [Agile, kapitel 7-12] hvor det er relevant.




\section{Planlægning af næste iteration}
En stor del af arbejdet i den tidligere iteration har fokuseret på at skabe et fornuftigt udviklingsmiljø, hvorefter denne iteration har fokuseret på at implementere den mest essentielle funktionalitet og skabe en simpel prototype. Næste iteration fokuserer dernæst på at udbygge kodebasen således at prototypen overholder flere af de funktionelle og ikke-funktionelle krav beskrevet tidligere i dette dokument.\\

Som del af læringsprocessen har vi erfaret, at nogle af vores estimater angående resourceforbrug i forbindelse med implementationen har været unøjagtige og misvisende. Vi har derfor et skærpet fokus på estimering i den kommende iteration, således at vi bedst muligt kan prioritere på baggrund af estimater, og dermed være i stand til at udnytte vores tidsmæssige ressourcer bedre.\\

%Beslut hvilke use cases I vil implementere i næste iteration (ved herunder at dele dem op i opgaver som kan estimeres). Redegør for hvordan I har planlagt, estimeret og evt. beregnet jeres hastighed (jf. velocity i [Agile]). Vedlæg en liste over opgaver som I planlægger at implementere i næste iteration.

Vores foreløbige estimater indikerer, at vi i løbet af næste iteration vil være i stand til at afdække de fleste funktionelle krav og vi vil derfor, som tidligere beskrevet, fokusere på implementation af disse. Idet projektet styres efter en agil udviklingsmodel, skal der dog tages højde for løbende ændring af estimater, og dermed eventuelt omprioritering.


\appendix
\section{Github repository}

{\color{blue}
\texttt{\url{{https://github.com/Ambrodji/SU-ERROR404}}}
{\color{black}
\section{Template}
\label{bilag:Template}

\lstinputlisting{OnlineTA_Template.fs}

\section{Unit-Test}
\label{bilag:Unit-Test}

\lstinputlisting{OnlineTA_unitTestTemplate.fsx}

\subsection{Manuel-Test}
\label{bilag:Manuel-Test}

Alle manuelle test herunder returnerede det vi havde forventet.

\begin{itemize}
  \item[] mono OnlineTA\_Template.exe kaldt uden funktion giver en string, som forventet.
  \item[] getQuestion kaldt uden argument giver en string, som forventet.
  \item[] Kaldes getQuestion med "h" som argument giver en string, som forventet.
  \item[] Kaldes getQuestion med et int tal kalder funktionen med argumentet, som forventet.
  \item[] Kaldes evalAnswer med argumenterne $(-1,500)$ giver en bool værdien false, som forventet.
  \item[] Kaldes evalAnswer med argumenterne $(0,0)$ giver en bool værdien true, som forventet.
  \item[] Kaldes evalAnswer med ("blarg","blarg") som argumenter, så giver den true. Dog er dette ikke relevant dette kun er en Template, og det evalAnswer tager højde for kan ændres fra spil til spil.
\end{itemize}



\end{document}