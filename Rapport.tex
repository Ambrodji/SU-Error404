\documentclass[12pt, a4paper]{article}
%-------------------------------------------------
%       PACKAGES
%-------------------------------------------------
\usepackage[utf8x]{inputenc}          % Allows the user to input accented characters directly from the keyboard
\usepackage[T1]{fontenc}              % Oriented to output, that is, what fonts to use for printing characters 
\usepackage[danish]{babel}            % The language, you are writing in - you should be able to change to english

\usepackage{caption}                  % You can create captions for your tables, pictures etc. AND YOU WANT TO DO THAT
\usepackage{wrapfig}                  % Allows in-line images if needed
\usepackage{hyperref}                 % Allows you to create hyperlinks within the document
\usepackage{url}                      % Enables typesetting of hyperlinks

\usepackage{amsmath,amsfonts,amssymb} % Math in your tex 
\usepackage{mathtools}                % Math in your tex 

\usepackage{csquotes}                 % Provides advanced facilities for inline and display quotations
\usepackage[titletoc]{appendix}       % Names appendices "Appendix A" instead of just A in Contents

\usepackage{pdfpages}                 % We can now import pdf files to our tex file - win!
\usepackage{graphicx}                 % We can now import pictures - uhlalalah!
\usepackage{float}                    % Pictures stay were i tell them

\usepackage{algorithm}                % Beautiful algorithms
\usepackage{algpseudocode}            % Beautiful pseudocode

\usepackage{natbib}                   % Bibliography stuff

\usepackage{enumitem}                 % Control the layout of enumerate, itemize and description
\usepackage{fancyhdr}                 % Fancy header and footer
\usepackage{booktabs}                 % Nice tables
\usepackage{changepage}               % Change the page layout in the middle of a document
\usepackage{tipa}                     % Fonts

\usepackage{listings}                 % Typesetting programs
\usepackage{color}                    % Color management
\usepackage{wallpaper}
\usepackage{palatino}

\setlength\parindent{0pt}             % Removes the indent on first line of each each section

%-------------------------------------------------
%       FSHARP LISTINGS
%-------------------------------------------------

% Kan måske erstates med \usepackage{clrscode3e}

\definecolor{bluekeywords}{rgb}{0.13,0.13,1}
\definecolor{greencomments}{rgb}{0,0.5,0}
\definecolor{turqusnumbers}{rgb}{0.17,0.57,0.69}
\definecolor{redstrings}{rgb}{0.5,0,0}

\lstdefinelanguage{FSharp}
                {morekeywords={let, new, match, with, rec, open, module, namespace, type, of, member, and, for, in, do, begin, end, fun, function, try, mutable, if, then, else},
    keywordstyle=\color{bluekeywords},
    sensitive=false,
    morecomment=[l][\color{greencomments}]{///},
    morecomment=[l][\color{greencomments}]{//},
    morecomment=[s][\color{greencomments}]{{(*}{*)}},
    morestring=[b]",
    stringstyle=\color{redstrings}
    }

\lstset{language=FSharp, breaklines = true, stepnumber = 1, firstnumber = 1, numbers = left}              % set \lstlisting to always parse text as F# code

%-------------------------------------------------
%       DOCUMENT CONFIGURATIONS
%-------------------------------------------------
\fancyhf{}
\hypersetup{colorlinks=false,hidelinks, citecolor=black, urlcolor=black}
\DeclarePairedDelimiter{\ceil}{\lceil}{\rceil} % Math in your tex 
\DeclarePairedDelimiter\floor{\lfloor}{\rfloor} % Math in your tex


%%  Begin document
%%  ==================================================================
\begin{document}

%   Forside
%   ==================================================================
    \includepdf[pages={-}]{forside.pdf}

\section{Problembeskrivelse}

\section{D1}

\subsection{Krav til applikation}
Dette afsnit har til formål at afdække omfanget af den endelige løsning, ved hjælp af en afgrænsning af de ønskede funktionelle og ikke-funktionelle krav, der stilles til den endelige applikation. Det er tiltænkt, at kravene skal udgøre et stabilt fundament for projektet, der kan lægge grund for senere \emph{design choices} og dermed bidrage til at gøre senere planlægning lettere. I den sammenhæng kan det ses som et redskab, der fungerer som en rettesnor for projektet, således at projektets retning og \emph{scope} lettere kan tilpasses i tilfælde af eventuelle designmæssige ændringer i løbet af den løbende process. 

\subsubsection{Funktionelle krav}
Funktionelle krav er en betegnelse der dækker over kravene til selve applikationens funktionalitet - Der fokuseres dermed på \emph{hvad} der skal kunne gøres, og ikke \emph{hvordan}. De opstillede funktionelle krav kan ses nedenfor:
\begin{itemize}
  \item \textbf{Dynamisk webplatform} - Applikationen skal fungere som en web-applikation, der kan tilgås fra moderne webbrowsere (Google Chrome, Safari, Firefox, osv)
  \item \textbf{Rolle-opdeling} - Det skal være muligt at oprette brugere i systemet, med en af følgende roller: \emph{administrator}, \emph{lærer} eller \emph{elev}. (Beskrivelse af de forskellige rollers tilgængelige funktionalitet beskrives løbende i efterfølgende krav)
  \item \textbf{Brugeradministration} - Administratorer skal være i stand til at oprette/redigere og slette brugere.
  \item \textbf{Indhold} - Det skal være muligt for elever at deltage i ``spil'', der består af en række spørgsmål eleverne skal svare på. Der skal som minimum være 1 spil tilgængeligt, som dels fungerer som et \emph{proof of concept}, og dels som en skabelon som lærere kan benytte sig af til oprettelse af nye spil
  \item \textbf{Spil skabelon} - Spillet skal kunne generere pseudo-tilfældige aritmetiske udtryk (med plus, minus, division og multiplikation) der skal evalueres af eleven. Svarenes korrekthed skal valideres i F\#.
  \item \textbf{Statistik} - Brugere med rollen ``Administrator'' eller ``lærer'' skal kunne se statistik for elevers deltagelse og resultater fra diverse spil.
  \item \textbf{Opdatering af indhold} - Brugere med rollen ``lærer'' skal være i stand til at uploade nye ``spil'' systemet.
  \item \textbf{Feedback} - Brugere med rollen ``lærer'' skal kunne give skriftlig feedback for et afsluttet spil til eleven der udført spillet. Elever skal kunne se en historik og spil de har deltaget i, samt eventuel feedback for disse.
  \item \textbf{Brugerinddeling} - Elever skal kunne knyttes til lærere, således at de får adgang til et katalog af spil der er bestemt af læreren.
\end{itemize}

\subsubsection{Ikke-funktionelle krav}
I modsætning til de funktionelle krav, beskæftiger ikke-funktionelle krav sig med \emph{hvordan} tingene skal gøres. Endvidere dækker de over krav i forbindelse med \emph{usability}. De opstillede ikke-funktionelle krav kan ses nedenfor:
\begin{itemize}
    \item \textbf{Effektivitet} - Systemet skal være skalerbart og fleksibelt i en sådan grad, at en gennemsnitlig bærbar computer med serverapplikationen kørende, skal kunne understøtte 100 samtidigt forbundne klienter og samtidigt have en svartid på under 2 sekunder per besvaret spørgsmål.
    \item \textbf{Tilgængelighed - Elev} - En gennemsnitlig bruger med rollen ``elev'' skal kunne gennemføre et ``spil'' korrekt efter 5 minutters introduktion til systemet.
    \item \textbf{Tilgængelighed - Administrator} - En gennemsnitlig bruger med rollen ``administrator'' skal være i stand til at oprette/slette 
    \item \textbf{Sværhedsgrad} - Det skal være muligt at definere sværhedsgrad for et givent spørgsmål, således at det er muligt at præsentere brugeren for et spil der gradvist bliver sværere og sværere.    
\end{itemize}

\subsection{Opsætning af udviklingsmiljø}
Vores udviklingsmiljø vil bestå af 3 primære aspekter, disse vil vi arbejde med for at få udviklet vores ønsket produkt.
\begin{itemize}
\item Fsharp - programmeringsmiljøet kommer til at foregå vha. Fsharp. Vi har arbejdet med det i et halvt år, hvilket gør det et meget attraktivt valg for os.
\item Github - Et websted hvor vi kan arbejde med vores opsætning af hjemmesiden samt holde alle vores Fsharp funktioner samlet et sted, hvor alle fra gruppen, nemt kan få adgang til filerne. Samt rette i dem. \href{https://github.com/Ambrodji/SU-MMJ}{Github}
\item Caddy og Go-Lang - Henholdsvis en webserver, og et sprog vi vil bruge til at bygge små applikationer der kan udføre requests og returnere dynamisk indhold. Det er tiltænkt, at de scripts der skrives i go-lang skal eksekvere F\# programmer.
\end{itemize}

\subsection{Udforskning af teknologien}

Vi har valgt at køre vores program fra en webside, og for at køre denne web platform har vi valgt at bruge Caddy. Caddy er en webserver som bruges til at modtage forespørgsler, køre de rigtige filer for hvad der bliver forespurgt, og så sende et svar retur til klient siden. Det giver altså brugeren adgang til at se hvad der er på siden, og manøvrere rund på den.\\

Vi har yderligere sat et Go-Lang script op, som kan tilgås via. [host:"den specificerede port"]. I øjeblikket printer dette script bare noget til siden. For at få lavet et godt, struktureret og dynamisk design på siden, har vi tænkt os at bruge HTML i forbindelse med CSS. Vi vil nok tage brug af Javascript for at håndtere klient sidens logik og Go-Lang til at kalde vores F\# kode, som ligger på serversiden. \\

Alle vores spil vil blive skrevet i F\#, hvilket giver bedst mening iforhold til at disse spil nok skal bruges i fremtiden af instruktorer som skal undervise i PoP/F\#. Dette giver os også en solid base for biblioteker vi kan arbejde med til at bygge spillet/ene op, såsom FsCheck.

\subsection{Planlægning af næste iteration}
Til næste iteration har vi planlagt at vi skal have designet op og køre på hjemmesiden, dog ikke det endelige(Så det kan tilpasses), samt implementere noget kode, så vi kan få en prototype lavet.
Vi vil også gerne have en database op og kører, og forbinde den med hjemmesiden, så vi længere ind i projektet, kan indsamle statistiker eller anden form for data.

\section{Design}

Webdesign
UML-diagram

\section{Afprøvning}

\section{Udviklingsmiljø}

\section{Diskussion af projektarbejdet}



\section*{Bilag}

\section{Oversigt over delafleveringer}
\label{bilag:delafleveringer}

\section{GitHub-oversigt}
\label{bilag:github}

\section{Reviews}
\label{bilag:reviews}

\end{document}
%%  ==================================================================
%%  End document
